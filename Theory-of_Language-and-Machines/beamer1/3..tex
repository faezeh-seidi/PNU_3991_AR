\documentclass[a4paper,9pt]{beamer}
\usepackage[latin1]{inputenc}
\usepackage{xcolor}
\usepackage{graphicx}
\usepackage{beamer themeshadow}
\usetheme{Berlin}
\usecolortheme[named= green]{structure}
\begin{document}
\begin{frame}
\it
1. Find the complexity of following algorithm.\\
\includegraphics[scale=0.8]{s}\\
SOLUTION: IF n = 1,then the result is x. Returning x takes unit time .IF n $ >$ 1, then the algorithm enters into the else part . This part contains a multipplication and a same algorithm repetition up to n-1 time .So, we can write $ T(n-1) +2 if n> 1.$\\
So, we can write.
\end{frame}
\begin{frame}
\begin{center}
\includegraphics[scale=0.8]{p}\\
\end{center}
\end{frame}
\begin{frame}
\it
2. Write the divide and conquer algorithm for merge sort and find its complexity.[Sri Venkateswara University 2008]\\
SOLUTION: In merge sort, the n element array is divided into tow halves each with n/2 elements (for n = even), or one with n/2 and the other with n/2 +1 elements. The two halves and merged. The algorithm for performing merge sort is as follows:
\end{frame}
\begin{frame}
\includegraphics[scale=0.5]{a}
\end{frame}
\begin{frame}
\it
The algorithm{ \small Merge (min, m, max)} is performed in O(n) .The algorithm {\small mergesort (int min, int max)} of n element becomes half when it is called recursively. Thus, the whole algo-rithm takes\\
\begin{center}
$T(n) \leq 2T\left(\dfrac{n}{2}\right) +cn $\\

\vspace{3mm}
$ \leq 2\left[ 2T\left( \dfrac{n}{2^{n}}\right) +c\dfrac{n}{2}\right] +cn $\\


\vspace{3mm}
$ T(n)\leq 2^{2}T\left( \dfrac{n}{2^{2}}\right) +2cn $\\
..................................\\
..................................\\
$ T(n)\leq 2^{i} T\left( \dfrac{n}{2^{i}}\right) +icn $

\vspace{3mm}
$ If 2^{i} = n, i=log_{2} n $

\vspace{3mm}
$ T(n) \leq 2^{i}T(1) + cn log_{2} n $\\
\end{center}
T(1) =1 (time required to sort a list of one element).\\
Thus, the complexity of merge sort is O(n $log_{2} n$).


\end{frame}
\end{document}