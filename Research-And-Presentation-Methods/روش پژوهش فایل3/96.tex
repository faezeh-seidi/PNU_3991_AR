\documentclass[a4paper]{book}
\begin{document}
\begin{flushleft}
\textbf{CHAPTER SEVEN}
\end{flushleft}
and without jargon. This advice is especially important in text-based asynchronous environments where the ability to clarify is especially important in text -based asynchronous environments where the ability to clarify is reduced. For example , the e-researcher may ask a question to which the participant provides a lengthy reply that completely misses the point of the question. In a face-to-face interviw, the researcher will know early on by the response that the question has been misunderstood, and can quickly clarify the misunderstanding. The e-researcher can reduce the odds of misunderstanding and gather useful data by adhering to following guidelines:\\

\vspace{2mm}
 Familiarize yourself with the guage and culture of the target participants.\\
 Using the communication software, pilot the interview question(s) to a few indi-viduals (two or three) who have characteristics similar to (or the same as ) the intended participants. Tanking these actions prior to the interview will help iden-tify problems with language usage and avoid afaux pas through insensitivity.\\
 Ask one question at time.\\

\vspace{2mm}
This may seem like common sense. But Patton (1987), for example, points out a mis-take that interviewers often make-putting several questions together into one ques-tion (sometimes referred to as a double-barreled question). Responding to double-barreled questions can be quite difficult, if not impossible. Consider the fol-lowing quesion:\\


\vspace{2mm} 
Do you agree that the Internet is a useful tool for data collection, or do you think it is most useful as an information tool?

\vspace{2mm}
This sample question illustrates three types of problems. First,it asks two questions: (1) Do you agree that the Internet is a useful tool for data callection, and (2) do you think it is most useful as an information tool? Morever, it might be difficult to answer if the participant does not feel that Internet is useful as either as tool for data collection or as an information tool. Alternatively, the paricipant might feel that it is useful for data collection,and (2) do you think it is most useful as an information tool? Moreover, it might be difficult to answer if the participant does not feel that the Internet is useful for data collection or as information tool. Alternatively, the participan might feel that useful for both data collection and as an information tool. Hence, the participant might have dif-ficulty answering this qestion given that there are two questions and an indication that an "either/or" response must be made between the two.\\
The second problem with this question is that it is not an open-ended question.\\
Questions that begin with " Do you " or " Would you " allow the participant to respond with " yes/no " statements and, as such. do not provide the researcher with much insight about why. A better question format for the semi-structured interview is open-ended.\\
Questions beginning with " What do you think about," " Tell me your opinion about," or " How do you feel about ," allow the participant to respond in their owen terms.\\
Finally, in the above sample, the question are leding-particularly the first question.A question asking if someone "agrees" is valu-laden.The e-researcher should be careful not word question as directive (whit words such as agree or dis-agree) to guard againts intluencing the participant to respond in a certain way. A bet-ter way to word this question is:\\

\vspace{2mm}
\begin{center}
What is you opinion about the usefulness of the Internet?
\end{center}


\end{document}