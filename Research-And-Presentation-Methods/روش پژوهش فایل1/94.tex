\documentclass[a4paper]{book}
\begin{document}
\textbf{CHAPTER SEVEN}
\begin{flushleft}
of facilitating a conversation, rather than only a sequential interview, is increased. Depending on the characterisics of the participants, the e-researcher shoud use a writing style that is more conversational and friendly academic.
\begin{center}

Sample Welcome Message

\end{center}
Hello [first name of participant]:

 \vspace*{5mm}
 Thak you for agreeing to participate in this interview.Further to my invitation letter, I would like to arrange adata to begin the interview.I was hoping we might begin Monday of next take longer than about 10 minutes each. I also thought we could cover one question every second day, but I am flexible and could make an alternate arrangement if this is inconvenient.
 
 \vspace{5mm}
 If you need to contact me other than by email, you can call me at [area code and phone number] during the day or [area code and phone number] during the evening.\\
 I look forward hearing form you!
 
 \vspace{5mm}
 Regards,
 
 \vspace{3mm}
 [first name of e-researcher]
 
 \vspace{3mm}
 \begin{flushright}
  We sugget the questions should not be presented tothe participant at this \\
  \end{flushright}
 point in the inthe interview process. The rationale for this is that ,typically, in unstructured and semi- structured inerviews, the questions change based on earlier responses.As the interview progresses, for example, the e-researcher may wish to build on previous comments or rephrase questions so that they conform to the language of the partici-pant.Moreover, during the interview, the e-researcher my discover an unexpected area that needs further exploration and is extremely relevant and enlightening to the topic.
 
 \vspace{5mm}
 \textbf{ASKING THE QUESTIONS}
 
 \vspace{3mm}
 Asking good inerview qustions is considered to be an art, because getting meaning-ful answers is often difficult and despends to great extent on the wording and tone of the questions. Fontana and Frey (1994) maintain that no matter how carefully we word the questions(s). ther is always a residue of ambiguity. This problem is magnified in text- based asynchronous interviews. One of the most effective ways of reducing this ambiguity is to use good questioning techniques, which in turn will enable the e-researcher to obtain more accurate information from the participants and garner useful.
 \end{flushleft}



















\end{document}
