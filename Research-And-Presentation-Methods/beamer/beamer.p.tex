\documentclass[a4paper]{beamer}
\usepackage[latin1]{inputenc}
\usepackage{xcolor}
\usepackage{graphicx}
\usepackage{beamer themeshadow}
\usetheme{Berlin}
\usecolortheme[named=yellow]{structure}
\begin{document}
\begin{frame}
\hspace*{0.1cm}\textbf{$|$}
\hspace*{0.1cm}\textbf{94}\\
of facilitating a conversation, rather than only a sequential interview, is increased. Depending on the characterisics of the participants, the e-researcher shoud use a writing style that is more conversational and friendly academic.
\begin{center}

Sample Welcome Message

\end{center}
Hello [first name of participant]:

 \vspace*{5mm}
 Thak you for agreeing to participate in this interview.Further to my invitation letter, I would like to arrange adata to begin the interview.I was hoping we might begin Monday of next take longer than about 10 minutes each. I also thought we could cover one question every second day, but I am flexible and could make an alternate arrangement if this is inconvenient.
 
 \vspace{5mm}
 If you need to contact me other than by email, you can call me at [area code and phone number] during the day or [area code and phone number] during the evening.\\
 I look forward hearing form you!
\end{frame}
\begin{frame}
\hspace*{0.1cm}\textbf{$|$}
\hspace*{0.1cm}\textbf{95}\\ 
\begin{flushright}
SEMEI-STRUCTURED AND UNSTRUCTURED INTERVIEWS
\end{flushright}
and meaningful data.No matter how much effort the researcher puts into the gues-tions, there is always the possibility of misinterpretation or misunderstanding by the participant.Thus, it is essential that the e-researcher create an atmosphere in which participants are trusting and feel confident enough to seek clarificartion.\\
Once a friendly tone has been established through welcome letter,it can be maintained by showing respect for the participants opinions through supportive acknowledgments of his or her responses. For example, you could begin questioning with the following email:\\
\end{frame}
\begin{frame}
\hspace*{0.1cm}\textbf{$|$}
\hspace*{0.1cm}\textbf{96}\\
and without jargon. This advice is especially important in text-based asynchronous environments where the ability to clarify is especially important in text -based asynchronous environments where the ability to clarify is reduced. For example , the e-researcher may ask a question to which the participant provides a lengthy reply that completely misses the point of the question. In a face-to-face interviw, the researcher will know early on by the response that the question has been misunderstood, and can quickly clarify the misunderstanding. The e-researcher can reduce the odds of misunderstanding and gather useful data by adhering to following guidelines:\\

\end{frame}
\end{document}