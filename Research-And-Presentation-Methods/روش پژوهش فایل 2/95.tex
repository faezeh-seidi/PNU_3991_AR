\documentclass[10pt]{book}
\begin{document}
\begin{flushright}
SEMEI-STRUCTURED AND UNSTRUCTURED INTERVIEWS
\end{flushright}
and meaningful data.No matter how much effort the researcher puts into the gues-tions, there is always the possibility of misinterpretation or misunderstanding by the participant.Thus, it is essential that the e-researcher create an atmosphere in which participants are trusting and feel confident enough to seek clarificartion.\\
Once a friendly tone has been established through welcome letter,it can be maintained by  showing respect for the participants opinions through supportive acknowledgments of his or her responses. For example, you could begin questioning with the following email:\\

\vspace{3mm}
Hello [first name of participant]:

\vspace{3mm}
Thank you again for agreein to participate in this study. I'd like to begin with the following question:

\vspace{3mm}
How does e-learning affect the way you design and teach courses?

\vspace{3mm}
If you have any questions or need further clarification, please let me know through email or phone me collect at [area code and phone number].

\vspace{4mm}
Regards,

\vspace{3mm}
[first name of e-researcher]

\vspace{3mm}
Once the participants has respondedtothe question, feedback such as "Thank you for that last response" acknowleges the response and indicates that youare attentive,interested,understanding, and respectful of what is being communicated.You should also ensure that the participant has finished sharing prior to moving on to the next question. Knowing when the participant is finised is trickier in a Nwt-based interview than in a face- to-face interview, due to the absence of nonverbal cues. The e-researchershould not assume that just because a posting is sent that the participant has finished saying what needs to be said.For example, the time required to compose the message with the necessary explanation may be longer than the participant anticipated, and, as such, may be cut short. Because the e-researcher cnnot "see" what is happen-ing, a follow-up message should always be sent asking "Do you have anyting else to add?" or "Do you think there is anything else I should know about this?" If the partic-ipant has anything further to say , this will provide an opportunity to xpand.\\
Once rapport and trust have been established , the e-researcher will need to focus on the questions. Obtaining accurate and meaningful over theNet requires carefully worded and articulater question .One way to reduce the residue of ambiguity is to use words that make sense to the paricipants. Cicourel (1964). who first wrote about this, noted that many meanings that may be clear to the researcher, may not make sense to participants. The main cause of this foible is that the researcher has not been sensitive to the participants context and worldview. According to Kvale (1996), to reduce the communicative ambiguity, the questions should be easy to understand, short,


\end{document}